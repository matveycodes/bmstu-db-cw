\chapter{Конструкторский раздел}

В данном разделе будет спроектирована база данных, которая требуется для реализации поставленной задачи, и описаны её ограничения. Также будет описана используемая ролевая модель.

\section{Диаграмма проектируемой базы данных}

На рисунке~\ref{img:db} представлена диаграмма проектируемой базы данных.

\includeimage
{db}
{f}
{H}
{\textwidth}
{Диаграмма проектируемой базы данных}

\section{Сущности проектируемой базы данных}

\subsection{Токен авторизации (таблица auth\_tokens)}

Сущность токена авторизации содержит следующие поля.

\begin{enumerate}
	\item user\_id --- уникальный идентификатор пользователя. Тип: UUID.
	\item value --- значение токена. Тип: строка.
	\item date\_expired --- дата окончания действия. Тип: дата со временем.
\end{enumerate}

\subsection{Сообщение в чате (таблица chat\_messages)}

Сущность сообщения в чате содержит следующие поля.

\begin{enumerate}
	\item id --- уникальный идентификатор сообщения. Тип: UUID.
	\item sender\_id --- уникальный идентификатор пользователя, отправившего сообщение. Тип: UUID.
	\item recipient\_id --- уникальный идентификатор пользователя, получившего сообщение. Тип: UUID.
	\item content --- текст сообщения. Тип: текст.
	\item date\_sent --- дата отправки сообщения. Тип: дата со временем.
\end{enumerate}

\subsection{СМС с кодом (таблица totp)}

Сущность СМС с кодом содержит следующие поля.

\begin{enumerate}
	\item code --- случайный код. Тип: целое число.
	\item date\_sent --- дата отправки кода. Тип: дата со временем.
	\item date\_used --- дата использования кода. Тип: дата со временем.
	\item phone --- номер телефона, на который был отправлен код. Тип: строка.
	\item signature --- уникальный токен. Тип: строка.
\end{enumerate}

\subsection{Настройка (таблица settings)}

Сущность настройки содержит следующие поля.

\begin{enumerate}
	\item name --- название. Тип: строка.
	\item value --- значение. Тип: текст.
\end{enumerate}

\subsection{Пользователь (таблица users)}

Сущность пользователя содержит следующие поля.

\begin{enumerate}
	\item id --- уникальный идентификатор пользователя. Тип: UUID.
	\item status --- статус пользователя. Тип: строка. Может принимать одно из трёх возможных значений: \enquote{pending} (ожидающий подтверждения возраста), \enquote{active} (активный) и \enquote{blocked} (заблокированный).
	\item date\_joined --- дата регистрации пользователя в системе. Тип: дата со временем.
	\item middle\_name --- отчество. Тип: строка.
	\item first\_name --- имя. Тип: строка.
	\item last\_name --- фамилия. Тип: строка.
	\item email --- адрес электронной почты. Тип: строка.
	\item phone --- номер телефона. Тип: строка.
	\item birthdate --- дата рождения. Тип: дата.
	\item role --- роль пользователя. Тип: строка. Может принимать одно из четырёх возможных значений: \enquote{customer} (клиент), \enquote{technician} (механик), \enquote{supporter} (сотрудник клиентской поддержки) и \enquote{admin} (администратор).
\end{enumerate}

\subsection{Аренда (таблица rentals)}

Сущность аренды содержит следующие поля.

\begin{enumerate}
	\item id --- уникальный идентификатор аренды. Тип: UUID.
	\item user\_id --- уникальный идентификатор пользователя. Тип: UUID.
	\item scooter\_id --- уникальный идентификатор пользователя. Тип: UUID.
	\item start\_price --- стоимость начала аренды в копейках. Тип: целое число.
	\item per\_minute\_price --- стоимость аренды за минуту в копейках. Тип: целое число.
	\item date\_started --- дата начала аренды. Тип: дата со временем.
	\item date\_finished --- дата завершения аренды. Тип: дата со временем.
\end{enumerate}

\subsection{Запись в истории перемещений (таблица pings)}

Сущность записи в истории перемещений содержит следующие поля.

\begin{enumerate}
	\item id --- уникальный идентификатор записи. Тип: UUID.
	\item scooter\_id --- уникальный идентификатор самоката, отправившего информацию. Тип: UUID.
	\item date --- дата создания записи. Тип: дата со временем.
	\item meta\_info --- информация о техническом состоянии электросамоката. Тип: JSON.
	\item location --- геолокация электросамоката. Тип: географические координаты.
	\item battery\_level --- уровень заряда батареи в процентах. Тип: целое число.
	\item lock\_state --- состояние замка. Тип: строка. Может принимать одно из двух возможных значений: \enquote{locked} (закрыт) и \enquote{unlocked} (открыт).
	\item lights\_state --- состояние фар. Тип: строка. Может принимать одно из двух возможных значений: \enquote{on} (включены) и \enquote{off} (выключены).
\end{enumerate}

\subsection{Зона ограничения скорости (таблица restricted\_zones)}

Сущность зоны ограничения скорости содержит следующие поля.

\begin{enumerate}
	\item id --- уникальный идентификатор зоны. Тип: UUID.
	\item polygon --- координаты зоны. Тип: массив географических координат.
	\item speed\_limit --- максимальная скорость. Тип: целое число.
\end{enumerate}

\subsection{Парковки (таблица parkings)}

Сущность парковки содержит следующие поля.

\begin{enumerate}
	\item id --- уникальный идентификатор парковки. Тип: UUID.
	\item location --- координаты парковки. Тип: географические координаты.
\end{enumerate}

\subsection{Бронирование (таблица bookings)}

Сущность бронирования содержит следующие поля.

\begin{enumerate}
	\item id --- уникальный идентификатор бронирования. Тип: UUID.
	\item user\_id --- уникальный идентификатор пользователя. Тип: UUID.
	\item scooter\_id --- уникальный идентификатор электросамоката. Тип: UUID.
	\item date\_started --- дата начала бронирования. Тип: дата со временем.
	\item date\_finished --- дата окончания бронирования. Тип: дата со временем.
\end{enumerate}

\subsection{Электросамокат (таблица scooters)}

Сущность электросамоката содержит следующие поля.

\begin{enumerate}
	\item id --- уникальный идентификатор самоката. Тип: UUID.
	\item model\_id --- уникальный идентификатор модели. Тип: UUID.
	\item status --- статус. Тип: строка. Может принимать одно из двух значений: \enquote{enabled} (активный) и \enquote{disabled} (деактивированный).
	\item image\_link --- ссылка на фотографию самоката. Тип: строка.
	\item number --- номер самоката. Тип: строка.
\end{enumerate}

\subsection{Модель электросамоката (таблица scooter\_models)}

Сущность модели электросамоката содержит следующие поля.

\begin{enumerate}
	\item id --- уникальный идентификатор модели. Тип: UUID.
	\item manufacturer\_id --- уникальный идентификатор производителя. Тип: UUID.
	\item title --- название модели. Тип: строка.
	\item single\_charge\_mileage --- пробег на одном заряде в километрах. Тип: целое число.
	\item weight --- масса электросамоката в килограммах. Тип: целое число.
	\item max\_speed --- максимальная скорость в километрах в час. Тип: целое число.
	\item max\_load --- максимальная нагрузка в килограммах. Тип: целое число.
	\item year --- год выпуска. Тип: целое число.
\end{enumerate}

\subsection{Производитель электросамоката (таблица scooter\_manufacturers)}

Сущность производителя электросамоката содержит следующие поля.

\begin{enumerate}
	\item id --- уникальный идентификатор производителя. Тип: UUID.
	\item title --- название. Тип: строка.
\end{enumerate}

\section{Ограничения целостности базы данных}

Для обеспечения целостности базы данных введены следующие ограничения.

\subsection{Таблица auth\_tokens}

Значение столбца user\_id не может быть пустым. Является внешним ключом: ссылается на столбец id таблицы users.

Значение столбца value должно быть уникальным в пределах таблицы. Не может быть пустым.

Значение столбца date\_expired не может быть пустым.

\subsection{Таблица chat\_messages}

Значение столбца id является первичным ключом. Не может быть пустым. По умолчанию заполняется случайным UUID.

Значение столбца sender\_id не может быть пустым. Является внешним ключом: ссылается на столбец id таблицы users.

Значение столбца recipient\_id не может быть пустым. Является внешним ключом: ссылается на столбец id таблицы users.

Значение столбца content не может быть пустым.

Значение столбца date\_sent не может быть пустым. По умолчанию заполняется отметкой текущего времени.

\subsection{Таблица totp}

Значение столбца code не может быть пустым.

Значение столбца date\_sent не может быть пустым. По умолчанию заполняется отметкой текущего времени.

Значение столбца text не может быть пустым.

Значение столбца signature должно быть уникальным в пределах таблицы. Не может быть пустым.

Значение столбца date\_used должно быть позже даты отправки (date\_sent).

\subsection{Таблица settings}

Значение столбца name должно быть уникальным в пределах таблицы. Не может быть пустым.

Значение столбца value не может быть пустым.

\subsection{Таблица users}

Значение столбца id является первичным ключом. Не может быть пустым. По умолчанию заполняется случайным UUID.

Значение столбца status может принимать одно из трёх значений: \enquote{pending}, \enquote{active}, \enquote{blocked}. Не может быть пустым. По умолчанию заполняется значением \enquote{pending}.

Значение столбца date\_joined не может быть пустым. По умолчанию заполняется отметкой текущего времени. Дата должна быть позже, чем дата рождения (birthdate).

Значение столбца phone должно быть уникальным в пределах таблицы. Не может быть пустым. Должно иметь формат +7NNNNNNNNNN, где N --- целое число от 0 до 9.

Значение столбца birthdate должно быть позже 1 января 1930 г.

Значение столбца role может принимать одно из четырёх возможных значений: \enquote{customer}, \enquote{technician}, \enquote{supporter}, \enquote{admin}. Не может быть пустым. По умолчанию заполняется значением \enquote{customer}.

\subsection{Таблица rentals}

Значение столбца id является первичным ключом. По умолчанию заполняется случайным UUID. Не может быть пустым.

Значение столбца user\_id не может быть пустым. Является внешним ключом: ссылается на столбец id таблицы users.

Значение столбца scooter\_id не может быть пустым. Является внешним ключом: ссылается на столбец id таблицы scooters.

Значение столбца start\_price не может быть пустым. Должно быть не меньше 0.

Значение столбца per\_minute\_price не может быть пустым. Должно быть не меньше 0.

Значение столбца date\_started не может быть пустым. По умолчанию заполняется отметкой текущего времени.

Значение столбца date\_finished должно быть позже, чем дата начала (date\_started).

\subsection{Таблица pings}

Значение столбца scooter\_id не может быть пустым. Является внешним ключом: ссылается на столбец id таблицы scooters.

Значение столбца date не может бысть пустым. По умолчанию заполняется отметкой текущего времени.

Значение столбца location не может быть пустым.

Значение столбца battery\_level не может быть пустым. Должно быть от 0 до 100 включительно.

Значение столбца lock\_state может принимать одно из двух значений: \enquote{locked} и \enquote{unlocked}. Не может быть пустым.

Значение столбца lights\_state может принимать одно из двух значений \enquote{on} и \enquote{off}. Не может быть пустым.

\subsection{Таблица restricted\_zones}

Значение столбца id является первичным ключом. Не может быть пустым. По умолчанию заполняется случайным UUID.

Значение столбца polygon не может быть пустым.

Значение столбца speed\_limit не может быть пустым. Должно быть не меньше 0.

\subsection{Таблица parkings}

Значение столбца id является первичным ключом. Не может быть пустым. По умолчанию заполняется случайным UUID.

Значение столбца location не может быть пустым.

\subsection{Таблица bookings}

Значение столбца id является первичным ключом. Не может быть пустым. По умолчанию заполняется случайным UUID.

Значение столбца user\_id не может быть пустым. Является внешним ключом: ссылается на столбец id таблицы users.

Значение столбца scooter\_id не может быть пустым. Является внешним ключом: ссылается на столбец id таблицы scooters.

Значение столбца date\_started не может быть пустым. По умолчанию заполняется отметкой текущего времени.

Значение столбца date\_finished не может быть пустым. Дата завершения должна быть позже, чем дата начала (date\_started).

\subsection{Таблица scooters}

Значение столбца id является первичным ключом. По умолчанию заполняется случайным UUID. Не может быть пустым.

Значение столбца model\_id не может быть пустым. Является внешним ключом: ссылается на столбец id таблицы scooter\_models.

Значение столбца status может принимать одно из двух значений: \enquote{enabled} и \enquote{disabled}. Не может быть пустым. По умолчанию заполняется значением \enquote{disabled}.

Значение столбца number не может быть пустым.

\subsection{Таблица scooter\_models}

Значение столбца id является первичным ключом. По умолчанию заполняется случайным UUID. Не может быть пустым.

Значение столбца manufacturer\_id не может быть пустым. Является внешним ключом: ссылается на столбец id таблицы scooter\_manufacturers.

Значение столбца title не может быть пустым.

Значение столбца single\_charge\_mileage не может быть пустым. Должно быть положительным целым числом.

Значение столбца weight не может быть пустым. Должно быть положительным целым числом.

Значение столбца max\_speed не может быть пустым. Должно быть положительным целым числом.

Значение столбца max\_load не может быть пустым. Должно быть положительным целым числом.

Значение столбца year не может быть пустым. Должно быть больше 2000.

\subsection{Таблица scooter\_manufacturers}

Значение столбца id является первичным ключом. По умолчанию заполняется случайным UUID. Не может быть пустым.

Значение столбца title не может быть пустым.

\section{Описание триггера}

На рисунке~\ref{img:trigger} представлена схема алгоритма работы триггера, который не позволяет удалять пользователей, которые имеют активные (незавершенные) аренды.

\includeimage
{trigger}
{f}
{H}
{.5\textwidth}
{Схема алгоритма работы триггера}

\section{Ролевая модель}

TODO

\section*{Вывод}

В данном разделе была спроектирована база данных для разрабатываемого приложения, а так же рассмотрены способы обеспечения ее целостности и ролевая модель.