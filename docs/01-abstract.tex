\chapter*{ВВЕДЕНИЕ}
\addcontentsline{toc}{chapter}{ВВЕДЕНИЕ}

С начала 2000-х годов отмечается появление шеринговых центров и развитие экономики совместного пользования за определённую сумму без права владения  (шеринг-экономики). За последнее десятилетие шеринг стал одной из основных идей, изменивших представление о потреблении.

Бурный рост мегаполисов предъявляет всё более строгие требования к городской мобильности. Кикшеринг-сервисы стали ответом на современные вызовы. Кикшеринг можно считать наиболее инновационной системой краткосрочной аренды транспортных средств. Массовое использование электросамокатов в качестве городского транспорта началось сравнительно недавно: в России сервисы проката электросамокатов появились в 2018 году. Сегодня арендовать электросамокаты можно в более чем 30 городах страны.

В связи с растущей популярностью подобных сервисов~\cite{popularity} всё чаще возникает проблема хранения операционных данных о поездках, пользователях, парке электросамокатов и многом другом.

Целью данной работы является проектирование и разработка базы данных для хранения данных сервиса краткосрочной аренды электросамокатов.

Для достижения данной цели необходимо решить следующие задачи:

\begin{itemize}
    \item проанализировать варианты представления данных и выбрать подходящий вариант для решения задачи;
    \item проанализировать системы управления базами данных и выбрать подходящую систему для хранения данных;
    \item спроектировать базу данных, описать её сущности и связи;
    \item реализовать интерфейс для доступа к базе данных;
    \item реализовать программное обеспечение, позволяющее взаимодействовать со спроектированной базой данных.
\end{itemize}