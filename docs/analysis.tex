\chapter{Аналитический раздел}

В данном разделе будет проведен анализ предметной области и существующих решений, будут выдвинуты требования к приложению, определены пользователи системы, формализованы хранимые сервисом данные, а также выбрана модель базы данных.

\section{Анализ предметной области}

Предметной областью поставленной задачи является операционная деятельность сервиса краткосрочной аренды электросамокатов в крупном российском городе.

Пользователями кикшеринг-сервиса могут стать лица, достигшие возраста 18~лет~[\ref{whoosh-reqs}][\ref{urent-reqs}][\ref{yago-reqs}]. Для регистрации необходим номер телефона и реквизиты банковской карты. Для отправки чеков может потребоваться адрес электронной почты.

Для поиска доступных для аренды электросамокатов пользователи используют специально разработанное программное обеспечение (в большинстве случаев --- мобильное приложение), где на карте отмечены свободные самокаты. В этом же приложении пользователи кикшеринг-сервиса выбирают самокат, а затем или бронируют, или сразу берут его в аренду.

Бронирование позволяет скрыть электросамокат с карты для других пользователей сервиса на определенное время. В большинстве кикшеринг-сервисов бронирование бесплатно и предусматривает возможность отмены без необходимости начинать аренду. При этом некоторые сервисы могут выдвигать дополнительные условия: например, находиться рядом с бронируемым самокатом.

Для начала аренды пользователи сканируют QR-код на корпусе электросамоката или вводят его номер в приложении.

Стоимость аренды складывается из двух составляющих: платы за старт и платы за каждую минуту поездки. В зависимости от спроса на самокаты на конкретной парковке цены могут быть скорректированы как в б\'{о}льшую, так и в меньшую сторону. Некоторые сервисы позволяют приобрести подписку, во время действия которой плата за начало поездок взиматься не будет. Перед началом аренды на банковской карте пользователя блокируется определенная сумма, которая возвращается после завершения поездки. По желанию пользователя и за дополнительную плату поездка может быть застрахована.

Во время поездки в мобильном приложении пользователю доступна информация о километраже, длительности и стоимости аренды. Также имеется возможность управлять самокатом: включать и выключать фары, открывать замок и выключать самокат.

Администрация кикшеринг-сервиса может устанавливать зоны, максимальная скорость движения в которых ограничена из соображений безопасности. Зачастую такие зоны вводятся в местах с активным пешеходным движением (например, парках). При попадании в одну из таких зон электросамокат автоматически сбрасывает скорость и не позволяет превысить установленный лимит.

Завершение аренды возможно на специальных парковках, отмеченных на карте в мобильном приложении. По завершении аренды у пользователя может быть запрошена фотография припаркованного самоката.

Для автоматизации учета информации о пользователях, поездках, самокатах, тарифах и т. д. разрабатывается специализированная информационная система. Использование данной системы возможно пользователями с различными уровнями доступа.

\section{Анализ существующих решений}

Для проведения анализа существующих решений среди имеющихся на российском рынке кикшеринг-сервисов были выбраны три наиболее популярных~[\ref{hype}]: Whoosh~[\ref{whoosh}], Юрент~[\ref{urent}] и Яндекс Go~[\ref{yago}]. Сравнение проводилось по следующим критериям.

\begin{enumerate}
	\item Максимальное количество одновременно арендованных электросамокатов на одном аккаунте.
	\item Возможность забронировать электросамокат, находясь на любом расстоянии до него.
	\item Необходимость делать фотографии припаркованного самоката после завершения поездки.
	\item Возможность поставить аренду на паузу по более низкой цене в минуту, чем во время активной аренды.
	\item Возможность посмотреть суммарный километраж всех поездок.
\end{enumerate}

Результат анализа представлен в таблице~\ref{tbl:comparison}.

\begin{table}[H]
	\caption{Сравнительный анализ существующих решений}
	\label{tbl:comparison}
	{\renewcommand{\arraystretch}{1.2}
		\begin{tabularx}{\textwidth}
			{
				| >{\raggedright\arraybackslash}X
				| >{\centering\arraybackslash}X
				| >{\centering\arraybackslash}X
				| >{\centering\arraybackslash}X |
			}
			\hline
			            & \textbf{Whoosh} & \textbf{Юрент} & \textbf{Яндекс Go} \\
			\hline
			Критерий №1 & не более 3      & не более 5     & не более 3         \\
			\hline
			Критерий №2 & $+$             & $-$            & $-$                \\
			\hline
			Критерий №3 & $-$             & $+$            & $+$                \\
			\hline
			Критерий №4 & $-$             & $-$            & $+$                \\
			\hline
			Критерий №5 & $+$             & $-$            & $-$                \\
			\hline
		\end{tabularx}}
\end{table}

Создаваемый в рамках курсовой работы сервис должен не просто быть на уровне с конкурентами, но и предоставлять пользователям дополнительный функционал. Конкурентными преимуществами станут:

\begin{itemize}
	\item возможность взять в аренду до 10 самокатов одновременно;
	\item возможность бронировать самокат с любого расстояния;
	\item отсутствие необходимости делать фотографию после завершения аренды.
\end{itemize}

\section{Требования к разрабатываемой базе данных и приложению}

Предметная область поставленной задачи является обширной и включает в себя множество понятий и связей между ними, поэтому были сформулированы следующие требования к разрабатываемой в рамках курсовой работы программе.

\begin{enumerate}
	\item Должен быть предоставлен функционал для регистрации и авторизации пользователей в системе.
	\item После регистрации пользователя в системе для него должен быть создан личный кабинет с основной информацией.
	\item Должен быть предоставлен функционал для получения списка парковок, зон ограничения максимальной скорости и доступных для аренды электросамокатов.
	\item Должен быть предоставлен функционал для бронирования электросамокатов.
	\item Должен быть предоставлен функционал для начала и завершения аренды, а также управления электросамокатом во время поездки.
	\item Должен быть предоставлен функционал для ограничения скорости электросамокатов в определенных зонах города.
	\item Должен быть предоставлен функционал для сохранения истории поездок с возможностью просмотра пользователем своих поездок.
\end{enumerate}

Также были сформулированы следующие допущения:

\begin{enumerate}
	\item Верификация номера телефона пользователя, равно как и отправка на него СМС-сообщений реализуется сторонним сервисом, предоставляющим программный интерфейс для взаимодействия (API). Реализация подобного функционала выходит за рамки курсовой работы по базам данных.
	\item Верификация адреса электронной почты пользователя путем отправки на него письма со ссылкой для подтверждения выходит за рамки курсовой работы по базам данных.
	\item Привязка банковской карты, оплата поездок и подписок производится сторонним платежным сервисом, предоставляющим программный интерфейс для взаимодействия (API). Реализация подобного функционала выходит за рамки курсовой работы по базам данных.
\end{enumerate}

\pagebreak
\section{Формализация данных}

На~рисунке~\ref{img:er} представлена ER-диаграмма сущностей проектируемой базы данных в нотации Чена, описывающая объекты предметной области и их взаимодействие.

\includeimage
{er}
{f}
{H}
{.95\textwidth}
{ER-диаграмма сущностей проектируемой базы данных}

База данных, проектируемая в ходе выполнения курсовой работы, включает в себя информацию о следующих объектах.

\begin{enumerate}
	\item Пользователь --- сущность, описывающая пользователя сервиса.
	\item Парковка --- сущность, описывающая городскую велопарковку или любое другое место, где можно оставить самокат и завершить аренду.
	\item Зона ограничения скорости --- сущность, описывающая часть города, где максимальная скорость движения ограничена из соображений безопасности.
	\item Электросамокат --- сущность, хранящая информацию о самокате, который пользователи могут брать в аренду.
	\item Запись в истории перемещений --- сущность, хранящая информацию о положении электросамоката в пространстве в определенный момент времени.
	\item Модель --- сущность, описывающая модель электросамоката.
	\item Производитель --- сущность, описывающая производителя электросамоката.
	\item Бронирование --- сущность, описывающая процесс резервирования пользователем конкретного самоката на определенный промежуток времени.
	\item Аренда --- сущность, описывающая процесс проката пользователем электросамоката с фиксированным на момент начала поездки тарифом.
	\item Настройка --- сущность, хранящая соответствие имени переменной и некоторого значения (например, стоимости страховки).
	\item Токен авторизации --- сущность, которая используется для авторизации запросов от пользователя к серверу для исключения передачи паролей.
	\item СМС с кодом --- сущность, хранящая информацию об отправленном на номер телефона пользователя сообщении с коротким кодом авторизации.
	\item Подписка --- сущность, описывающая подписку, с которой пользователь может бесплатно начинать поездки.
\end{enumerate}

\section{Формализация пользователей приложения}

Ролевая модель используется для реализации системы безопасности сервера базы данных и позволяет разрешать или запрещать тем или иным группам пользователей работу с объектами базы данных.

В рамках поставленной задачи были выделены следующие роли.

\begin{enumerate}
	\item Гость --- пользователь, не авторизовавшийся в системе. Имеет минимальный уровень доступа: ему доступен только список парковок и зон ограничения скорости.
	\item Клиент. Может находиться в одном из двух состояний:
	      \begin{enumerate}
		      \item не подтвердивший свой возраст;
		      \item подтвердивший свой возраст (активный клиент).
	      \end{enumerate}
	      Так как полноценное пользование сервисом возможно только совершеннолетними лицами, после регистрации все клиенты находятся в неподтвержденном состоянии. После указания своего возраста и в случае, если тот больше или равен 18-ти, статус клиента меняется на активный.
	\item Механик. Имеет доступ к списку самокатов, требующих обслуживания, и их местоположению.
	\item Электросамокат. Имеет возможность создавать записи в истории перемещений.
	\item Администратор. Имеет максимальный уровень доступа ко всем данным сервиса.
\end{enumerate}

\pagebreak
\section{Диаграмма вариантов использования}

На рисунках~\ref{img:unauth-use-case}--\ref{img:admin-use-case} представлены диаграммы вариантов использования проектируемой информационной системы в соответствии с формализацией пользователей приложения.

\includeimage
{unauth-use-case}
{f}
{H}
{.4\textwidth}
{Диаграмма вариантов использования системы гостем}

\includeimage
{pending-client-use-case}
{f}
{H}
{.6\textwidth}
{Диаграмма вариантов использования системы не подтвердившим свой возраст клиентом}

\includeimage
{active-client-use-case}
{f}
{H}
{\textwidth}
{Диаграмма вариантов использования системы активным клиентом}

\includeimage
{technician-use-case}
{f}
{H}
{.6\textwidth}
{Диаграмма вариантов использования системы механиком}

\includeimage
{scooter-use-case}
{f}
{H}
{.4\textwidth}
{Диаграмма вариантов использования системы электросамокатом}

\includeimage
{admin-use-case}
{f}
{H}
{.8\textwidth}
{Диаграмма вариантов использования системы администратором}

\section{Анализ существующих баз данных}

По организации и способу хранения данных все базы данных можно разделить на две группы: реляционные и нереляционные. Реляционные базы данных в свою очередь делятся на строковые и колоночные, а нереляционные --- на графовые, документные и базы данных типа \enquote{ключ-значение}.

\subsection{Реляционные базы данных}

Реляционные базы данных основываются на реляционной модели данных. Данные в таких базах организованы в виде набора таблиц, состоящих из столбцов и строк. В таблицах хранится информация об объектах, представленных в базе данных. Такие базы данных удобно использовать для хорошо структурированных данных.

\subsubsection{Строковые базы данных}

Строковыми называются базы данных, записи которых в памяти представлены построчно. Строковые базы данных используются в транзакционных системах. Для таких систем характерно большое количество коротких транзакций с операциями вставки, обновления и удаления данных.

\subsubsection{Колоночные базы данных}

Колоночными называются базы данных, записи которых в памяти представляются по столбцам. Колоночные базы данных используются в аналитических системах. Такие системы характеризуются низким объемом транзакций, а запросы к ним зачастую сложны и включают в себя агрегацию.

\subsection{Нереляционные базы данных}

Нереляционная база данных --- это база данных, в которой в отличие от большинства традиционных систем баз данных не используется табличная схема строк и столбцов. В этих базах данных применяется модель хранения, оптимизированная под конкретные требования типа хранимых данных.

\subsubsection{Базы данных \enquote{ключ-значение}}

В базах данных \enquote{ключ-значение} данные хранятся как совокупность пар ключ-значение, где ключ служит уникальным идентификатором. Такие базы данных удобно использовать для хранения и обработки разных по типу и содержанию данных, их легко масштабировать. Однако такой тип баз данных не подходит для работы со сложными и связанными друг с другом данными.

\subsubsection{Документные базы данных}

Документные базы данных --- это тип нереляционных баз данных, предназначенный для хранения и запроса данных в виде документов в формате, подобном JSON. Такие базы данных позволяют хранить и запрашивать данные в базе данных с помощью той же документной модели, которая используются в коде приложения. Документные базы данных хорошо подходят для быстрой разработки систем и сервисов, работающих с по-разному структурированными данными. Они легко масштабируются и меняют структуру при необходимости. Однако такие базы данных теряют свою эффективность при решении задач, в которых требуется работа с множеством связанных объектов.

\subsubsection{Графовые базы данных}

Графовые базы данных --- это тип нереляционных баз данных, предназначенный для хранения взаимосвязей между сущностями и навигации по ним. Для хранения сущностей используются узлы, а для хранения их взаимосвязей --- ребра. В таких базах отсутствуют ограничения на количество и тип взаимосвязей, которые может иметь узел. Графовые базы данных используются для решения задач, имеющих сложные взаимосвязи между данными. При незначительном количестве связей и больших объемах данных графовые базы данных демонстрируют значительно более низкую производительность.

Для решения поставленной задачи была выбрана реляционная база данных с построчным хранением данных, так как:

\begin{itemize}
	\item задача предполагает хранение структурированных и связанных между собой данных;
	\item задача предполагает постоянное добавление и изменение данных;
	\item задача предполагает быструю отзывчивость на запросы пользователя;
	\item задача не предполагает выполнения сложных аналитических запросов.
\end{itemize}

\section*{Вывод}

В данном разделе была спроектирована ролевая модель системы, формализованы хранимые данные и их связь между собой, построены соответствующие диаграммы. Также был проведен анализ существующих на рынке решений, который позволил понять, какие особенности стоит добавить в разрабатываемый проект. Был осуществлен выбор модели базы данных.