\chapter{Конструкторский раздел}

В данном разделе будет спроектирована база данных, которая требуется для реализации поставленной задачи, описаны ее ограничения. Также будет описан и спроектирован триггер и используемая ролевая модель.

\section{Диаграмма проектируемой базы данных}

На рисунке~\ref{img:db} представлена диаграмма проектируемой базы данных.

\includeimage
{db}
{f}
{H}
{\textwidth}
{Диаграмма проектируемой базы данных}

\pagebreak
\section{Сущности проектируемой базы данных}

\subsection{Токен авторизации (таблица auth\_tokens)}

Сущность токена авторизации содержит следующие поля.

\begin{enumerate}
	\item user\_id --- уникальный идентификатор пользователя. Тип: UUID.
	\item value --- уникальное значение токена. Тип: строка.
	\item date\_expired --- дата окончания действия. Тип: дата со временем.
\end{enumerate}

\subsection{СМС с кодом (таблица totp)}

Сущность СМС с кодом содержит следующие поля.

\begin{enumerate}
	\item code --- случайный код. Тип: целое число.
	\item date\_sent --- дата отправки кода. Тип: дата со временем.
	\item date\_used --- дата использования кода. Тип: дата со временем.
	\item phone --- номер телефона, на который был отправлен код. Тип: строка.
	\item signature --- уникальная подпись сообщения. Тип: строка.
\end{enumerate}

\subsection{Настройка (таблица settings)}

Сущность настройки содержит следующие поля.

\begin{enumerate}
	\item name --- название. Тип: строка.
	\item value --- значение. Тип: текст.
\end{enumerate}

\subsection{Пользователь (таблица users)}

Сущность пользователя содержит следующие поля.

\begin{enumerate}
	\item id --- уникальный идентификатор пользователя. Тип: UUID.
	\item status --- статус пользователя. Тип: строка.
	\item date\_joined --- дата регистрации пользователя в системе. Тип: дата со
	      временем.
	\item middle\_name --- отчество. Тип: строка.
	\item first\_name --- имя. Тип: строка.
	\item last\_name --- фамилия. Тип: строка.
	\item email --- адрес электронной почты. Тип: регистронезависимая строка.
	\item phone --- номер телефона. Тип: строка.
	\item birthdate --- дата рождения. Тип: дата.
	\item role --- роль пользователя. Тип: строка.
\end{enumerate}

\subsection{Аренда (таблица rentals)}

Сущность аренды содержит следующие поля.

\begin{enumerate}
	\item id --- уникальный идентификатор аренды. Тип: UUID.
	\item user\_id --- уникальный идентификатор пользователя, который начал аренду. Тип:
	      UUID.
	\item scooter\_id --- уникальный идентификатор арендованного электросамоката. Тип:
	      UUID.
	\item start\_price --- стоимость начала аренды в копейках. Тип: целое число.
	\item per\_minute\_price --- стоимость минуты аренды в копейках. Тип: целое число.
	\item date\_started --- дата начала аренды. Тип: дата со временем.
	\item date\_finished --- дата завершения аренды. Тип: дата со временем.
\end{enumerate}

\subsection{Запись в истории перемещений (таблица pings)}

Сущность записи в истории перемещений содержит следующие поля.

\begin{enumerate}
	\item scooter\_id --- уникальный идентификатор электросамоката, отправившего
	      информацию. Тип: UUID.
	\item date --- дата создания записи. Тип: дата со временем.
	\item meta\_info --- дополнительная информация о техническом состоянии
	      электросамоката. Тип: JSON.
	\item location --- местоположение электросамоката. Тип: географические координаты.
	\item battery\_level --- уровень заряда батареи в количестве процентов. Тип: целое
	      число.
	\item lock\_state --- состояние замка. Тип: строка.
	\item lights\_state --- состояние фар. Тип: строка.
\end{enumerate}

\subsection{Зона ограничения скорости (таблица restricted\_zones)}

Сущность зоны ограничения скорости содержит следующие поля.

\begin{enumerate}
	\item id --- уникальный идентификатор зоны. Тип: UUID.
	\item polygon --- координаты зоны. Тип: массив географических координат.
	\item speed\_limit --- максимальная скорость в километрах в час. Тип: целое число.
\end{enumerate}

\subsection{Парковки (таблица parkings)}

Сущность парковки содержит следующие поля.

\begin{enumerate}
	\item id --- уникальный идентификатор парковки. Тип: UUID.
	\item location --- местоположение парковки. Тип: географические координаты.
\end{enumerate}

\subsection{Бронирование (таблица bookings)}

Сущность бронирования содержит следующие поля.

\begin{enumerate}
	\item id --- уникальный идентификатор бронирования. Тип: UUID.
	\item user\_id --- уникальный идентификатор пользователя, создавшего бронирование.
	      Тип: UUID.
	\item scooter\_id --- уникальный идентификатор забронированного электросамоката. Тип:
	      UUID.
	\item date\_started --- дата начала бронирования. Тип: дата со временем.
	\item date\_finished --- дата окончания бронирования. Тип: дата со временем.
\end{enumerate}

\subsection{Электросамокат (таблица scooters)}

Сущность электросамоката содержит следующие поля.

\begin{enumerate}
	\item id --- уникальный идентификатор самоката. Тип: UUID.
	\item model\_id --- уникальный идентификатор модели электросамоката. Тип: UUID.
	\item status --- статус электросамоката. Тип: строка.
	\item number --- номер самоката. Тип: строка.
\end{enumerate}

\subsection{Модель электросамоката (таблица scooter\_models)}

Сущность модели электросамоката содержит следующие поля.

\begin{enumerate}
	\item id --- уникальный идентификатор модели. Тип: UUID.
	\item manufacturer\_id --- уникальный идентификатор производителя модели. Тип: UUID.
	\item title --- название модели. Тип: строка.
	\item single\_charge\_mileage --- пробег на одном заряде в километрах. Тип: целое
	      число.
	\item weight --- масса электросамоката в килограммах. Тип: целое число.
	\item max\_speed --- максимальная скорость в километрах в час. Тип: целое число.
	\item max\_load --- максимальная нагрузка в килограммах. Тип: целое число.
	\item year --- год выпуска. Тип: целое число.
\end{enumerate}

\subsection{Производитель электросамоката (таблица scooter\_manufacturers)}

Сущность производителя электросамоката содержит следующие поля.

\begin{enumerate}
	\item id --- уникальный идентификатор производителя. Тип: UUID.
	\item title --- название. Тип: строка.
\end{enumerate}

\subsection{Подписка (таблица subscriptions)}

Сущность подписки содержит следующие поля.

\begin{enumerate}
	\item id --- уникальный идентификатор подписки. Тип: UUID.
	\item title --- название подписки. Тип: строка.
	\item duration --- длительность действия подписки в секундах. Тип: целое число.
	\item price --- стоимость подписки в копейках. Тип: целое число.
\end{enumerate}

\subsection{Купленная подписка (таблица purchased\_subscriptions)}

Сущность купленной подписки содержит следующие поля.

\begin{enumerate}
	\item subscription\_id --- уникальный идентификатор купленной подписки. Тип: UUID.
	\item user\_id --- уникальный идентификатор пользователя, купившего подписку. Тип:
	      UUID.
	\item date\_started --- дата начала действия подписки. Тип: дата со временем.
	\item date\_finished --- дата окончания действия подписки. Тип: дата со временем.
	\item date\_purchased --- дата покупки подписки. Тип: дата со временем.
\end{enumerate}

\section{Ограничения целостности базы данных}

Для обеспечения целостности базы данных вводятся различные ограничения на формат хранимых данных.

\subsection{Таблица auth\_tokens}

Значение столбца user\_id не может быть пустым. Является внешним ключом: ссылается на столбец id таблицы users.

Значение столбца value должно быть уникальным в пределах таблицы. Не может быть пустым.

Значение столбца date\_expired не может быть пустым.

\subsection{Таблица totp}

Значение столбца code не может быть пустым.

Значение столбца date\_sent не может быть пустым. По умолчанию заполняется отметкой текущего времени.

Значение столбца text не может быть пустым.

Значение столбца signature должно быть уникальным в пределах таблицы. Не может быть пустым.

Значение столбца date\_used должно быть позже даты отправки~(date\_sent).

\subsection{Таблица settings}

Значение столбца name должно быть уникальным в пределах таблицы. Не может быть пустым.

Значение столбца value не может быть пустым.

\subsection{Таблица users}

Значение столбца id является первичным ключом. Не может быть пустым. По умолчанию заполняется случайным UUID.

Столбец status может принимать одно из трех возможных значений: \enquote{pending} (ожидающий подтверждения возраста), \enquote{active} (активный), \enquote{blocked} (заблокированный). Не может быть пустым. По умолчанию заполняется значением \enquote{pending}.

Значение столбца date\_joined не может быть пустым. По умолчанию заполняется отметкой текущего времени. Дата должна быть позже, чем дата рождения (birthdate).

Значение столбца phone должно быть уникальным в пределах таблицы. Не может быть пустым. Должно иметь формат 7$NNNNNNNNNN$, где $N$~---~целое число от 0 до 9.

Значение столбца birthdate должно быть позже 1 января 1930 г.

Столбец role может принимать одно из трех возможных значений: \enquote{customer} (клиент), \enquote{technician} (механик) и \enquote{admin} (администратор). Не может быть пустым. По умолчанию заполняется значением \enquote{customer}.

\subsection{Таблица rentals}

Значение столбца id является первичным ключом. По умолчанию заполняется случайным UUID. Не может быть пустым.

Значение столбца user\_id не может быть пустым. Является внешним ключом: ссылается на столбец id таблицы users.

Значение столбца scooter\_id не может быть пустым. Является внешним ключом: ссылается на столбец id таблицы scooters.

Значение столбца start\_price не может быть пустым. Должно быть не меньше 0.

Значение столбца per\_minute\_price не может быть пустым. Должно быть не меньше 0.

Значение столбца date\_started не может быть пустым. По умолчанию заполняется отметкой текущего времени.

Значение столбца date\_finished должно быть позже, чем дата начала~(date\_started).

\subsection{Таблица pings}

Значение столбца scooter\_id не может быть пустым. Является внешним ключом: ссылается на столбец id таблицы scooters.

Значение столбца date не может быть пустым. По умолчанию заполняется отметкой текущего времени.

Значение столбца location не может быть пустым.

Значение столбца battery\_level не может быть пустым. Должно быть от~0 до~100 включительно.

Столбец lock\_state может принимать одно из двух возможных значений: \enquote{locked} (замок закрыт) и \enquote{unlocked} (замок открыт). Не может быть пустым.

Столбец lights\_state может принимать одно из двух возможных значений: \enquote{on} (фары включены) и \enquote{off} (фары выключены). Не может быть пустым.

\subsection{Таблица restricted\_zones}

Значение столбца id является первичным ключом. Не может быть пустым. По умолчанию заполняется случайным UUID.

Значение столбца polygon не может быть пустым.

Значение столбца speed\_limit не может быть пустым. Должно быть не меньше 0.

\subsection{Таблица parkings}

Значение столбца id является первичным ключом. Не может быть пустым. По умолчанию заполняется случайным UUID.

Значение столбца location не может быть пустым.

\subsection{Таблица bookings}

Значение столбца id является первичным ключом. Не может быть пустым. По умолчанию заполняется случайным UUID.

Значение столбца user\_id не может быть пустым. Является внешним ключом: ссылается на столбец id таблицы users.

Значение столбца scooter\_id не может быть пустым. Является внешним ключом: ссылается на столбец id таблицы scooters.

Значение столбца date\_started не может быть пустым. По умолчанию заполняется отметкой текущего времени.

Значение столбца date\_finished не может быть пустым. Дата завершения должна быть позже, чем дата начала (date\_started).

\subsection{Таблица scooters}

Значение столбца id является первичным ключом. По умолчанию заполняется случайным UUID. Не может быть пустым.

Значение столбца model\_id не может быть пустым. Является внешним ключом: ссылается на столбец id таблицы scooter\_models.

Столбец status может принимать одно из двух возможных значений: \enquote{enabled} (доступен) и \enquote{disabled} (недоступен). Не может быть пустым. По умолчанию заполняется значением \enquote{disabled}.

Значение столбца number должно быть уникальным в пределах таблицы. Не может быть пустым.

\subsection{Таблица scooter\_models}

Значение столбца id является первичным ключом. По умолчанию заполняется случайным UUID. Не может быть пустым.

Значение столбца manufacturer\_id не может быть пустым. Является внешним ключом: ссылается на столбец id таблицы scooter\_manufacturers.

Значение столбца title не может быть пустым.

Значение столбца single\_charge\_mileage не может быть пустым. Должно быть положительным целым числом.

Значение столбца weight не может быть пустым. Должно быть положительным целым числом.

Значение столбца max\_speed не может быть пустым. Должно быть положительным целым числом.

Значение столбца max\_load не может быть пустым. Должно быть положительным целым числом.

Значение столбца year не может быть пустым. Должно быть больше~2000.

\subsection{Таблица scooter\_manufacturers}

Значение столбца id является первичным ключом. По умолчанию заполняется случайным UUID. Не может быть пустым.

Значение столбца title не может быть пустым.

\subsection{Таблица subscriptions}

Значение столбца id является первичным ключом. По умолчанию заполняется случайным UUID. Не может быть пустым.

Значение столбца title не может быть пустым.

Значение столбца duration не может быть пустым. Должно быть положительным целым числом.

Значение столбца price не может быть пустым. Должно быть не меньше~0.

\subsection{Таблица purchased\_subscriptions}

Значение столбца subscription\_id не может быть пустым. Является внешним ключом: ссылается на столбец id таблицы subscriptions.

Значение столбца user\_id не может быть пустым. Является внешним ключом: ссылается на столбец id таблицы users.

Значение столбца date\_started не может быть пустым. По умолчанию заполняется отметкой текущего времени.

Значение столбца date\_finished не может быть пустым. Дата должна быть позже, чем дата начала (date\_started).

Значение столбца date\_purchased не может быть пустым. По умолчанию заполняется отметкой текущего времени.

\pagebreak
\section{Описание триггера}

На рисунке~\ref{img:trigger} представлена схема алгоритма работы триггера, который не позволяет удалять пользователей, которые имеют активные (незавершенные) аренды.

\includeimage
{trigger}
{f}
{H}
{.5\textwidth}
{Схема алгоритма работы триггера}

\pagebreak
\section{Ролевая модель}

Спроектированная в соответствии с формализацией пользователей сервиса ролевая модель доступа к таблицам разрабатываемой базы данных представлена в таблицах~\ref{tbl:guest-policy}-\ref{tbl:admin-policy}.

\begin{table}[H]
	\begin{threeparttable}[b]
		\caption{Политика доступа гостя}
		\label{tbl:guest-policy}
		{\renewcommand{\arraystretch}{1.2}
			\begin{tabularx}{\textwidth}
				{
					| >{\raggedright\arraybackslash}X
					| >{\centering\arraybackslash}c
					| >{\centering\arraybackslash}c
					| >{\centering\arraybackslash}c
					| >{\centering\arraybackslash}c |
				}
				\hline
				                         & \textbf{Чтение} & \textbf{Вставка} & \textbf{Обновление} & \textbf{Удаление} \\
				\hline
				auth\_tokens             & $-$             & $-$              & $-$                 & $-$               \\
				\hline
				bookings                 & $-$             & $-$              & $-$                 & $-$               \\
				\hline
				parkings                 & $+$             & $-$              & $-$                 & $-$               \\
				\hline
				pings                    & $-$             & $-$              & $-$                 & $-$               \\
				\hline
				purchased\_subscriptions & $-$             & $-$              & $-$                 & $-$               \\
				\hline
				rentals                  & $-$             & $-$              & $-$                 & $-$               \\
				\hline
				restricted\_zones        & $+$             & $-$              & $-$                 & $-$               \\
				\hline
				scooter\_manufacturers   & $-$             & $-$              & $-$                 & $-$               \\
				\hline
				scooter\_models          & $-$             & $-$              & $-$                 & $-$               \\
				\hline
				scooters                 & $-$             & $-$              & $-$                 & $-$               \\
				\hline
				settings                 & $-$             & $-$              & $-$                 & $-$               \\
				\hline
				subscriptions            & $-$             & $-$              & $-$                 & $-$               \\
				\hline
				totp                     & $-$             & $-$              & $-$                 & $-$               \\
				\hline
				users                    & $-$             & $-$              & $-$                 & $-$               \\
				\hline
			\end{tabularx}}
	\end{threeparttable}
\end{table}

\begin{table}[H]
	\begin{threeparttable}[b]
		\caption{Политика доступа не подтвердившего свой возраст клиента}
		\label{tbl:pending-client-policy}
		{\renewcommand{\arraystretch}{1.2}
			\begin{tabularx}{\textwidth}
				{
					| >{\raggedright\arraybackslash}X
					| >{\centering\arraybackslash}c
					| >{\centering\arraybackslash}c
					| >{\centering\arraybackslash}c
					| >{\centering\arraybackslash}c |
				}
				\hline
				                         & \textbf{Чтение}            & \textbf{Вставка} & \textbf{Обновление}        & \textbf{Удаление} \\
				\hline
				auth\_tokens             & $-$                        & $-$              & $-$                        & $-$               \\
				\hline
				bookings                 & $-$                        & $-$              & $-$                        & $-$               \\
				\hline
				parkings                 & $+$                        & $-$              & $-$                        & $-$               \\
				\hline
				pings                    & $-$                        & $-$              & $-$                        & $-$               \\
				\hline
				purchased\_subscriptions & $-$                        & $-$              & $-$                        & $-$               \\
				\hline
				rentals                  & $-$                        & $-$              & $-$                        & $-$               \\
				\hline
				restricted\_zones        & $+$                        & $-$              & $-$                        & $-$               \\
				\hline
				scooter\_manufacturers   & $-$                        & $-$              & $-$                        & $-$               \\
				\hline
				scooter\_models          & $-$                        & $-$              & $-$                        & $-$               \\
				\hline
				scooters                 & $-$                        & $-$              & $-$                        & $-$               \\
				\hline
				settings                 & $-$                        & $-$              & $-$                        & $-$               \\
				\hline
				subscriptions            & $-$                        & $-$              & $-$                        & $-$               \\
				\hline
				totp                     & $-$                        & $-$              & $-$                        & $-$               \\
				\hline
				users                    & $+$\rlap{\rlap{\tnote{1}}} & $-$              & $+$\rlap{\rlap{\tnote{1}}} & $-$               \\
				\hline
			\end{tabularx}}
		\begin{tablenotes}
			\item [1] Операция доступна только для строки, хранящей информацию об этом клиенте.
		\end{tablenotes}
	\end{threeparttable}
\end{table}

\begin{table}[H]
	\begin{threeparttable}[b]
		\caption{Политика доступа активного клиента}
		\label{tbl:active-client-policy}
		{\renewcommand{\arraystretch}{1.2}
			\begin{tabularx}{\textwidth}
				{
					| >{\raggedright\arraybackslash}X
					| >{\centering\arraybackslash}c
					| >{\centering\arraybackslash}c
					| >{\centering\arraybackslash}c
					| >{\centering\arraybackslash}c |
				}
				\hline
				                         & \textbf{Чтение}     & \textbf{Вставка} & \textbf{Обновление} & \textbf{Удаление} \\
				\hline
				auth\_tokens             & $-$                 & $-$              & $-$                 & $-$               \\
				\hline
				bookings                 & $+$\rlap{\tnote{1}} & $+$              & $-$                 & $-$               \\
				\hline
				parkings                 & $+$                 & $-$              & $-$                 & $-$               \\
				\hline
				pings                    & $+$\rlap{\tnote{2}} & $-$              & $-$                 & $-$               \\
				\hline
				purchased\_subscriptions & $+$\rlap{\tnote{3}} & $+$              & $-$                 & $-$               \\
				\hline
				rentals                  & $+$\rlap{\tnote{4}} & $+$              & $+$\rlap{\tnote{4}} & $-$               \\
				\hline
				restricted\_zones        & $+$                 & $-$              & $-$                 & $-$               \\
				\hline
				scooter\_manufacturers   & $+$                 & $-$              & $-$                 & $-$               \\
				\hline
				scooter\_models          & $+$                 & $-$              & $-$                 & $-$               \\
				\hline
				scooters                 & $+$\rlap{\tnote{5}} & $-$              & $-$                 & $-$               \\
				\hline
				settings                 & $-$                 & $-$              & $-$                 & $-$               \\
				\hline
				subscriptions            & $+$                 & $-$              & $-$                 & $-$               \\
				\hline
				totp                     & $-$                 & $-$              & $-$                 & $-$               \\
				\hline
				users                    & $+$\rlap{\tnote{6}} & $-$              & $+$\rlap{\tnote{6}} & $-$               \\
				\hline
			\end{tabularx}}
		\begin{tablenotes}
			\item [1] Операция доступна только для бронирований, которые были созданы этим клиентом.
			\item [2] Операция доступна только для последних записей электросамокатов, имеющих положительный уровень заряда батареи.
			\item [3] Операция доступна только для подписок, купленных этим клиентом.
			\item [4] Операция доступна только для аренд, которые были начаты этим клиентом.
			\item [5] Операция доступна только для самокатов в статусе \enquote{enabled}.
			\item [6] Операция доступна только для строки, хранящей информацию об этом клиенте.
		\end{tablenotes}
	\end{threeparttable}
\end{table}

\begin{table}[H]
	\begin{threeparttable}[b]
		\caption{Политика доступа механика}
		\label{tbl:technician-policy}
		{\renewcommand{\arraystretch}{1.2}
			\begin{tabularx}{\textwidth}
				{
					| >{\raggedright\arraybackslash}X
					| >{\centering\arraybackslash}c
					| >{\centering\arraybackslash}c
					| >{\centering\arraybackslash}c
					| >{\centering\arraybackslash}c |
				}
				\hline
				                         & \textbf{Чтение}     & \textbf{Вставка} & \textbf{Обновление} & \textbf{Удаление} \\
				\hline
				auth\_tokens             & $-$                 & $-$              & $-$                 & $-$               \\
				\hline
				bookings                 & $-$                 & $-$              & $-$                 & $-$               \\
				\hline
				parkings                 & $+$                 & $-$              & $-$                 & $-$               \\
				\hline
				pings                    & $+$\rlap{\tnote{1}} & $-$              & $-$                 & $-$               \\
				\hline
				purchased\_subscriptions & $-$                 & $-$              & $-$                 & $-$               \\
				\hline
				rentals                  & $-$                 & $-$              & $-$                 & $-$               \\
				\hline
				restricted\_zones        & $+$                 & $-$              & $-$                 & $-$               \\
				\hline
				scooter\_manufacturers   & $+$                 & $-$              & $-$                 & $-$               \\
				\hline
				scooter\_models          & $+$                 & $-$              & $-$                 & $-$               \\
				\hline
				scooters                 & $+$                 & $-$              & $-$                 & $-$               \\
				\hline
				settings                 & $-$                 & $-$              & $-$                 & $-$               \\
				\hline
				subscriptions            & $-$                 & $-$              & $-$                 & $-$               \\
				\hline
				totp                     & $-$                 & $-$              & $-$                 & $-$               \\
				\hline
				users                    & $+$\rlap{\tnote{2}} & $-$              & $+$\rlap{\tnote{2}} & $-$               \\
				\hline
			\end{tabularx}}
		\begin{tablenotes}
			\item [1] Операция доступна только для последних записей электросамокатов, имеющих нулевой уровень заряда батареи (разряженных).
			\item [2] Операция доступна только для строки, хранящей информацию об этом механике.
		\end{tablenotes}
	\end{threeparttable}
\end{table}

\begin{table}[H]
	\begin{threeparttable}[b]
		\caption{Политика доступа электросамоката}
		\label{tbl:scooter-policy}
		{\renewcommand{\arraystretch}{1.2}
			\begin{tabularx}{\textwidth}
				{
					| >{\raggedright\arraybackslash}X
					| >{\centering\arraybackslash}c
					| >{\centering\arraybackslash}c
					| >{\centering\arraybackslash}c
					| >{\centering\arraybackslash}c |
				}
				\hline
				                         & \textbf{Чтение} & \textbf{Вставка} & \textbf{Обновление} & \textbf{Удаление} \\
				\hline
				auth\_tokens             & $-$             & $-$              & $-$                 & $-$               \\
				\hline
				bookings                 & $-$             & $-$              & $-$                 & $-$               \\
				\hline
				parkings                 & $-$             & $-$              & $-$                 & $-$               \\
				\hline
				pings                    & $-$             & $+$              & $-$                 & $-$               \\
				\hline
				purchased\_subscriptions & $-$             & $-$              & $-$                 & $-$               \\
				\hline
				rentals                  & $-$             & $-$              & $-$                 & $-$               \\
				\hline
				restricted\_zones        & $-$             & $-$              & $-$                 & $-$               \\
				\hline
				scooter\_manufacturers   & $-$             & $-$              & $-$                 & $-$               \\
				\hline
				scooter\_models          & $-$             & $-$              & $-$                 & $-$               \\
				\hline
				scooters                 & $-$             & $-$              & $-$                 & $-$               \\
				\hline
				settings                 & $-$             & $-$              & $-$                 & $-$               \\
				\hline
				subscriptions            & $-$             & $-$              & $-$                 & $-$               \\
				\hline
				totp                     & $-$             & $-$              & $-$                 & $-$               \\
				\hline
				users                    & $-$             & $-$              & $-$                 & $-$               \\
				\hline
			\end{tabularx}}
	\end{threeparttable}
\end{table}

\begin{table}[H]
	\begin{threeparttable}[b]
		\caption{Политика доступа администратора}
		\label{tbl:admin-policy}
		{\renewcommand{\arraystretch}{1.2}
			\begin{tabularx}{\textwidth}
				{
					| >{\raggedright\arraybackslash}X
					| >{\centering\arraybackslash}c
					| >{\centering\arraybackslash}c
					| >{\centering\arraybackslash}c
					| >{\centering\arraybackslash}c |
				}
				\hline
				                         & \textbf{Чтение} & \textbf{Вставка} & \textbf{Обновление} & \textbf{Удаление} \\
				\hline
				auth\_tokens             & $+$             & $+$              & $+$                 & $+$               \\
				\hline
				bookings                 & $+$             & $+$              & $+$                 & $+$               \\
				\hline
				parkings                 & $+$             & $+$              & $+$                 & $+$               \\
				\hline
				pings                    & $+$             & $+$              & $+$                 & $+$               \\
				\hline
				purchased\_subscriptions & $+$             & $+$              & $+$                 & $+$               \\
				\hline
				rentals                  & $+$             & $+$              & $+$                 & $+$               \\
				\hline
				restricted\_zones        & $+$             & $+$              & $+$                 & $+$               \\
				\hline
				scooter\_manufacturers   & $+$             & $+$              & $+$                 & $+$               \\
				\hline
				scooter\_models          & $+$             & $+$              & $+$                 & $+$               \\
				\hline
				scooters                 & $+$             & $+$              & $+$                 & $+$               \\
				\hline
				settings                 & $+$             & $+$              & $+$                 & $+$               \\
				\hline
				subscriptions            & $+$             & $+$              & $+$                 & $+$               \\
				\hline
				totp                     & $+$             & $+$              & $+$                 & $+$               \\
				\hline
				users                    & $+$             & $+$              & $+$                 & $+$               \\
				\hline
			\end{tabularx}}
	\end{threeparttable}
\end{table}

\section*{Вывод}

В данном разделе была спроектирована база данных для разрабатываемого приложения, рассмотрены способы обеспечения ее целостности, спроектирован триггер и ролевая модель.