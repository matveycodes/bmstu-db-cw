\chapter{Аналитический раздел}

В данном разделе будут выдвинуты требования к приложению, определены пользователи системы, формализованы хранимые сервисом данные. Также будет проведен анализ существующих решений и выбрана модель базы данных.

\section{Анализ предметной области}

TODO

\section{Анализ существующих решений}

Среди уже имеющихся сервисов кикшеринга были выбраны три наиболее популярных: Whoosh, Юрент и Яндекс Go~\cite{hype}. Сравнение проводилось по следующим критериям.

\begin{enumerate}
    \item Количество одновременных аренд разных самокатов, которое может начать пользователь с одного аккаунта.
    \item Возможность забронировать электросамокат, находясь на любом расстоянии до него.
    \item Динамическое ценообразование, зависящее от спроса на электросамокаты на конкретных парковках.
    \item Возможность поставить аренду на паузу по более низкой цене в минуту, чем во время активной аренды.
    \item Наличие тарифа <<пока не сядет>>, предполагающего более низкую стоимость аренды почти разряженного самоката.
    \item Наличие тарифа, предполагающего фиксированную стоимость поездки до заранее установленной конечной точки.
\end{enumerate}

Результат сравнения представлен в таблице \ref{tbl:comparison}.

\begin{table}[H]
    \caption{Сравнение существующих решений}
    \label{tbl:comparison}
    {\renewcommand{\arraystretch}{1.2}
    \begin{tabularx}{\textwidth} 
        {
            | >{\raggedright\arraybackslash}X 
            | >{\centering\arraybackslash}X 
            | >{\centering\arraybackslash}X 
            | >{\centering\arraybackslash}X |
        }
        \hline
        \textbf{Критерий} & \textbf{Whoosh} & \textbf{Юрент} & \textbf{Яндекс Go}\\
        \hline
        Количество одновременных аренд & до 3 & до 5 & до 3\\
        \hline
        Бронирование с любого расстояния & $+$ & $-$ & $-$\\
        \hline
        Динамическое ценообразование & $+$ & $+$ & $+$\\
        \hline
        Пауза по сниженной цене & $-$ & $-$ & $+$\\
        \hline
        Тариф <<пока не сядет>> & $-$ & $+$ & $-$\\
        \hline
        Фиксированный тариф & $-$ & $-$ & $+$\\
        \hline
    \end{tabularx}}
\end{table}

Создаваемый в рамках курсовой работы сервис должен не просто быть на уровне с конкурентами, но и предоставлять пользователям дополнительный функционал. Конкурентными преимуществами станут:

\begin{itemize}
    \item TODO
    \item TODO
    \item TODO
\end{itemize}

\section{Требования к разрабатываемой базе данных и приложению}

TODO

\section{Формализация данных}

TODO + ER-диаграмма сущностей проектируемой базы данных в нотации Чена

\section{Формализация пользователей приложения}

TODO

\section{Диаграмма вариантов использования}

TODO

\section{Анализ существующих баз данных}

По организации и способу хранения данных все базы данных можно разделить на две группы: реляционные и нереляционные. Реляционные базы данных в свою очередь делятся на строковые и колоночные, а нереляционные --- на графовые, документные и базы данных типа <<ключ-значение>>.

\subsection{Реляционные базы данных}

Реляционные базы данных основываются на реляционной модели данных. Данные в таких базах организованы в виде набора таблиц, состоящих из столбцов и строк. В таблицах хранится информация об объектах, представленных в базе данных. Такие базы данных удобно использовать для хорошо структурированных данных.

\subsubsection{Строковые базы данных}

Строковыми базами данных называются базы данных, записи которых в памяти представлены построчно. Строковые базы данных используются в транзакционных системах. Для таких систем характерно большое количество коротких транзакций с операциями вставки, обновления и удаления данных.

\subsubsection{Колоночные базы данных}

Колоночными базами данных называются базы данных, записи которых в памяти представляются по столбцам. Колоночные базы данных используются в аналитических системах. Такие системы характеризуются низким объемом транзакций, а запросы к ним зачастую сложны и включают в себя агрегацию.

\subsection{Нереляционные базы данных}

Нереляционная база данных --- это база данных, в которой в отличие от большинства традиционных систем баз данных не используется табличная схема строк и столбцов. В этих базах данных применяется модель хранения, оптимизированная под конкретные требования типа хранимых данных.

\subsubsection{Базы данных <<ключ-значение>>}

В базах данных <<ключ-значение>> данные хранятся как совокупность пар ключ-значение, где ключ служит уникальным идентификатором. Такие базы данных удобно использовать для хранения и обработки разных по типу и содержанию данных, их легко масштабировать. Однако такой тип баз данных не подходит для работы со сложными и связанными друг с другом данными.

\subsubsection{Документные базы данных}

Документные базы данных --- это тип нереляционных баз данных, предназначенный для хранения и запроса данных в виде документов в формате, подобном JSON. Такие базы данных позволяют хранить и запрашивать данные в базе данных с помощью той же документной модели, которая используются в коде приложения. Документные базы данных хорошо подходят для быстрой разработки систем и сервисов, работающих с по-разному структурированными данными. Они легко масштабируются и меняют структуру при необходимости. Однако такие базы данных теряют свою эффективность при решении задач, в которых требуется работа с множеством связанных объектов.

\subsubsection{Графовые базы данных}

Графовые базы данных --- это тип нереляционных баз данных, предназначенный для хранения взаимосвязей между сущностями и навигации по ним. Для хранения сущностей используются узлы, а для хранения их взаимосвязей --- ребра. В таких базах отсутствуют ограничения на количество и тип взаимосвязей, которые может иметь узел. Графовые базы данных используются для решении задач, имеющих сложные взаимосвязи между данными. При незначительном количестве связей и больших объемах данных графовые базы данных демонстрируют значительно более низкую производительность.

Для решения поставленной задачи была выбрана реляционная база данных с построчным хранением данных, так как:

\begin{itemize}
    \item задача предполагает хранение структурированных и связанных данных;
    \item задача предполагает постоянное добавление и изменение данных;
    \item задача предполагает быструю отзывчивость на запросы пользователя;
    \item задача не предполагает выполнения сложных аналитических запросов.
\end{itemize}

\section*{Вывод}

В данном разделе были выделены ролевые модели системы, конкретизированы хранимые данные и их связь между собой, построены соответствующие диаграммы. Также был проведен анализ существующих на рынке решений, который позволил понять, какие особенности стоит добавить в разрабатываемый проект. Был осуществлен выбор модели базы данных.