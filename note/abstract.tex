\chapter*{ВВЕДЕНИЕ}
\addcontentsline{toc}{chapter}{ВВЕДЕНИЕ}

Бурный рост мегаполисов предъявляет все более строгие требования к городской мобильности. Ответом на современные вызовы стали сервисы кикшеринга (англ. kicksharing), в которых пользователи могут арендовать электрические самокаты, велосипеды или другие транспортные средства на короткий период времени.

Основной отличительной чертой подобных сервисов является возможность аренды транспорта на несколько минут или часов, что делает их удобными для передвижения по городу на короткие расстояния. Самокаты на электрической тяге (электросамокаты) становятся все более популярными в городах по всему миру, так как они представляют собой экологически чистый и удобный способ передвижения.

Массовое использование электросамокатов в качестве городского транспорта началось сравнительно недавно: в России сервисы проката появились в 2018 году. Сегодня арендовать электросамокаты можно в более чем 100 городах страны~[\ref{urent-cities}].

В связи с растущей популярностью подобных сервисов~[\ref{popularity}] все чаще возникает проблема хранения операционных данных о поездках, пользователях, парке электросамокатов и многом другом.

Целью данной работы является проектирование и разработка информационной системы сервиса краткосрочной аренды городских электросамокатов.

Для достижения данной цели необходимо решить следующие задачи:

\begin{itemize}
	\item формализовать данные и пользователей приложения, сформулировать требования к разрабатываемой базе данных и приложению, выбрать модель базы данных;
	\item спроектировать базу данных, описать ее сущности и связи;
	\item реализовать программное обеспечение, позволяющее взаимодействовать со спроектированной базой данных;
	\item провести исследование зависимости времени выполнения запроса от объема данных и наличия или отсутствия индексации.
\end{itemize}