\chapter{Технологический раздел}

В данном разделе будет выбрана СУБД, описаны средства реализации приложения, создание базы данных, триггера, ролей с последующим выделением прав, наполнение базы, а также спроектирован пользовательский интерфейс.

\section{Выбор средств реализации}

В настоящее время существует множество систем управления базами данных, работающих по реляционной модели. Среди самых распространенных~[\ref{dmbs}] выделяют Oracle, MySQL, Microsoft SQL Server и PostgreSQL. В качестве системы управления базами данных была выбрана PostgreSQL~[\ref{postgres}], так как данная СУБД является свободным программным обеспечением (OSS) и предоставляется бесплатно.

В качестве языка программирования для написания серверной части был выбран TypeScript, поскольку он имеет все необходимые возможности для решения поставленной задачи и, в отличие от JavaScript, имеет строгую типизацию. Для данного языка также разработано множество библиотек, в том числе для взаимодействия с базами данных.

В ходе решения поставленной задачи были использованы дополнительные библиотеки, расширяющие базовый функционал языка TypeScript: koa~[\ref{koa}] --- для маршрутизации, обработки входящих и исходящих запросов, pg-promise~[\ref{pg-promise}] и sqlstring~[\ref{sqlstring}] --- для взаимодействия с базой данных.

Графический пользовательский интерфейс (GUI) представляет собой одностраничное приложение (SPA), написанное на TypeScript и React~[\ref{react}] при помощи библиотеки компонентов Chakra UI~[\ref{chakra}].

Для одновременного запуска всех компонентов сервиса используется Docker~[\ref{docker}] и система управления развертыванием Docker Compose. Это позволит разворачивать сервис на разных операционных системах без внесения изменений в исходный код.

\section{Архитектура сервиса}

Разрабатываемый в рамках курсовой работы сервис состоит из следующих взаимосвязанных компонентов:

\begin{itemize}
    \item клиентская часть (фронтенд);
    \item серверная часть (бэкенд);
    \item база данных.
\end{itemize}

Клиентская часть обменивается данными с серверной посредством REST API~[\ref{rest}]. Данные передаются в текстовом формате JSON, так как и фронтенд, и бэкенд написаны на диалекте JavaScript, для которого JSON является нативным способом представления данных.

Серверная часть сервиса взаимодействует с базой данных посредством коннекторов, позволяющих выполнять запросы к базе данных на языке программирования, используемом для разработки приложения.

\section{Методы тестирования}

При тестировании серверной части сервиса использовались как модульные, так и интеграционные тесты. Для их реализации была использована специализированная библиотека jest~[\ref{jest}].

Тестирование клиентской части производилось вручную.

\section{Детали реализации}

Скрипт создания таблиц базы данных, включая создание индексов, органичений, триггеров и описанных выше ролей, приведен в приложениях \ref{appendix:init}.

\section{Наполнение базы данных}

Некоторые таблицы базы данных были предварительно заполнены данными, похожими на реальные. Это упростило тестирование и позволит облегчить демонстрацию работы сервиса. Рассмотрим, каким образом осуществлялась генерация данных.

Данные для таблицы parkings со списком парковок города Москвы были взяты из OpenStreetMap~[\ref{osm}] --- открытого проекта картографических данных --- при помощи Overpass-запроса. Его код приведен в приложении \ref{appendix:overpass}. Итоговое количество парковок --- 2371.

Таблицы настроек settings и подписок subscriptions были заполнены данными, полученными из опыта взаимодействия с аналогичными кикшеринг-сервисами, в частности Whoosh и Юрент.

Зоны ограничения скорости для таблицы restricted\_zones были размечены вручную.

Список электросамокатов, хранящийся в таблице scooters, был сгенерирован случайным образом с ограничением на уникальность номера и имеет длину в 2000 записей.

Для каждого электросамоката из таблицы scooters была сгенерирована одна запись в истории перемещений со случайными параметрами. Местоположение самоката было подобрано таким образом, чтобы он оказался на одной из парковок, сгенерированных на предыдущем шаге.

\section{Интерфейс доступа к базе данных}

Для доступа к базе данных было разработано одностраничное веб-приложение.

На рисунке \ref{img:guest} представлен внешний вид приложения для неавторизованного пользователя, который имеет возможность посмотреть на карте парковки (отмечены иконками \enquote{P}) и зоны ограничения скорости (изображены как полупрозрачные многоугольники), а также авторизоваться или зарегистрироваться, нажав на кнопку \enquote{войти}.

\includeimage
{guest}
{f}
{h}
{\textwidth}
{Внешний вид веб-приложения для неавторизованного пользователя}

На рисунке \ref{img:active} изображен внешний вид приложения для активного клиента. На карте отмечены самокаты, доступные для аренды. Для удобства использования множество самокатов в одном месте кластеризуется. В левой части экрана клиент может отслеживать активные бронирования и поездки, управлять самокатом и ходом аренды, запрашивать историю поездок и список доступных к покупке подписок (рисунок \ref{img:subscriptions}).

\includeimage
{active}
{f}
{h}
{\textwidth}
{Внешний вид веб-приложения для активного клиента}

\includeimage
{subscriptions}
{f}
{h}
{\textwidth}
{Доступные подписки}

На рисунке \ref{img:active-history} изображен список поездок клиента, который тот может запросить в приложении, нажав на кнопку \enquote{История поездок}.

\includeimage
{active-history}
{f}
{h}
{\textwidth}
{История поездок}

На рисунке \ref{img:technician} изображен внешний вид приложения для механика. На карте отмечены самокаты, требующие обслуживания (замены или зарядки батареи).

\includeimage
{technician}
{f}
{h}
{\textwidth}
{Внешний вид веб-приложения для механика}

На рисунке \ref{img:admin} изображен внешний вид приложения для администратора. В левой части экрана представлены значения действующих на текущий момент настроек сервиса. Также есть возможность посмотреть список всех пользователей (рисунок \ref{img:admin-users}).

\includeimage
{admin}
{f}
{h}
{\textwidth}
{Внешний вид веб-приложения для администратора}

\includeimage
{admin-users}
{f}
{h}
{\textwidth}
{Список пользователей, доступный администратору сервиса}