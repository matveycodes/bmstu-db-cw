\chapter{Конструкторский раздел}

В данном разделе будет спроектирована база данных, которая требуется для реализации поставленной задачи, и описаны ее ограничения. Также будет описана используемая ролевая модель.

\section{Диаграмма проектируемой базы данных}

На рисунке~\ref{img:db} представлена диаграмма проектируемой базы данных.

\includeimage
    {db}
    {f}
    {H}
    {\textwidth}
    {Диаграмма проектируемой базы данных}

\section{Сущности проектируемой базы данных}

\subsection{Пользователь (таблица users)}

Сущность пользователя содержит следующие поля.

\begin{enumerate}
    \item id --- уникальный идентификатор пользователя. Тип: UUID.
    \item status --- статус пользователя. Тип: строка. Может принимать одно из трех возможных значений: <<pending>> (ожидающий подтверждения), <<active>> (активный) и <<blocked>> (заблокированный).
    \item date\_joined --- дата регистрации пользователя в системе. Тип: дата со временем.
    \item middle\_name --- отчество. Тип: строка.
    \item first\_name --- имя. Тип: строка.
    \item last\_name --- фамилия. Тип: строка.
    \item email --- адрес электронной почты. Тип: строка.
    \item phone --- номер телефона. Тип: строка.
    \item birthdate --- дата рождения. Тип: дата.
\end{enumerate}

\subsection{Сотрудник клиентской поддержки (таблица supports)}

Сущность сотрудника клиентской поддержки содержит те же поля, что и сущность пользователя, за исключением даты рождения (birthdate), даты регистрации (date\_joined) и статуса (status).

\subsection{Механик (таблица technicians)}

Сущность механика содержит те же поля, что и сущность пользователя, за исключением даты рождения (birthdate), даты регистрации (date\_joined) и статуса (status).

\subsection{Администратор (таблица admins)}

Сущность администратора содержит те же поля, что и сущность пользователя, за исключением даты рождения (birthdate), даты регистрации (date\_joined) и статуса (status).

\subsection{Токен авторизации (таблица auth\_tokens)}

Сущность токена авторизации содержит следующие поля.

\begin{enumerate}
    \item user\_id --- уникальный идентификатор пользователя. Тип: UUID.
    \item value --- хэш. Тип: строка.
    \item date\_expired --- дата окончания действия. Тип: дата со временем.
\end{enumerate}

\subsection{Сообщение в чате (таблица chat\_messages)}

Сущность сообщения в чате содержит следующие поля.

\begin{enumerate}
    \item id --- уникальный идентификатор сообщения. Тип: UUID.
    \item user\_id --- уникальный идентификатор пользователя, отправившего сообщение. Тип: UUID.
    \item support\_id --- уникальный идентификатор сотрудника клиентской поддержки. Тип: UUID.
    \item message --- текст сообщения. Тип: текст.
    \item date --- дата отправки сообщения. Тип: дата со временем.
\end{enumerate}

\subsection{Бронирование (таблица bookings)}

Сущность бронирования содержит следующие поля.

\begin{enumerate}
    \item id --- уникальный идентификатор бронирования. Тип: UUID.
    \item user\_id --- уникальный идентификатор пользователя. Тип: UUID.
    \item scooter\_id --- уникальный идентификатор электросамоката. Тип: UUID.
    \item date\_started --- дата начала бронирования. Тип: дата со временем.
    \item date\_finished --- дата окончания бронирования. Тип: дата со временем.
\end{enumerate}

\subsection{Аренда (таблица rentals)}

Сущность аренды содержит следующие поля.

\begin{enumerate}
    \item id --- уникальный идентификатор аренды. Тип: UUID.
    \item user\_id --- уникальный идентификатор пользователя. Тип: UUID.
    \item scooter\_id --- уникальный идентификатор пользователя. Тип: UUID.
    \item start\_price --- стоимость начала аренды в копейках. Тип: число.
    \item per\_minute\_price --- стоимость аренды за минуту в копейках. Тип: число.
    \item date\_started --- дата начала аренды. Тип: дата со временем.
    \item date\_finished --- дата завершения аренды. Тип: дата со временем.
\end{enumerate}

\subsection{Зона ограничения скорости (таблица restricted\_zones)}

Сущность зоны ограничения скорости содержит следующие поля.

\begin{enumerate}
    \item id --- уникальный идентификатор зоны. Тип: UUID. 
    \item polygon --- координаты зоны. Тип: Geometry.
    \item speed\_limit --- максимальная скорость. Тип: число.
\end{enumerate}

\subsection{Парковки (таблица parkings)}

Сущность парковки содержит следующие поля.

\begin{enumerate}
    \item id --- уникальный идентификатор парковки. Тип: UUID.
    \item location --- координаты парковки. Тип: Geography.
\end{enumerate}

\subsection{Настройка (таблица settings)}

Сущность настройки содержит следующие поля.

\begin{enumerate}
    \item name --- название. Тип: строка.
    \item value --- значение. Тип: текст.
\end{enumerate}

\subsection{Электросамокат (таблица scooters)}

Сущность электросамоката содержит следующие поля.

\begin{enumerate}
    \item id --- уникальный идентификатор самоката. Тип: UUID.
    \item model\_id --- уникальный идентификатор модели. Тип: UUID.
    \item status --- статус. Тип: строка. Может принимать одно из двух значений: <<enabled>> (активный) и <<disabled>> (деактивированный).
    \item image\_link --- ссылка на фотографию самоката. Тип: строка.
    \item number --- номер самоката. Тип: строка.
\end{enumerate}

\subsection{Модель электросамоката (таблица scooter\_models)}

Сущность модели электросамоката содержит следующие поля.

\begin{enumerate}
    \item id --- уникальный идентификатор модели. Тип: UUID.
    \item manufacturer\_id --- уникальный идентификатор производителя. Тип: UUID.
    \item title --- название модели. Тип: строка.
    \item single\_charge\_mileage --- пробег на одном заряде в километрах. Тип: число.
    \item weight --- масса электросамоката. Тип: число.
    \item max\_speed --- максимальная скорость в километрах в час. Тип: число.
    \item max\_load --- максимальная нагрузка в килограммах. Тип: число.
    \item year --- год выпуска. Тип: число.
\end{enumerate}

\subsection{Производитель электросамоката (таблица scooter\_manufacturers)}

Сущность производителя электросамоката содержит следующие поля.

\begin{enumerate}
    \item id --- уникальный идентификатор производителя. Тип: UUID.
    \item title --- название. Тип: строка.
\end{enumerate}

\subsection{Запись в истории перемещений (таблица pings)}

Сущность записи в истории перемещений содержит следующие поля.

\begin{enumerate}
    \item id --- уникальный идентификатор записи. Тип: UUID. 
    \item scooter\_id --- уникальный идентификатор самоката, отправившего информацию. Тип: UUID. 
    \item date --- дата создания записи. Тип: дата со временем. 
    \item meta\_info --- информация о техническом состоянии электросамоката. Тип: JSON. 
    \item location --- геолокация электросамоката. Тип: Geography. 
    \item battery\_level --- уровень заряда батареи. Тип: число. 
    \item lock\_state --- состояние замка. Тип: строка. Может принимать одно из двух возможных значений: <<locked>> (закрыт) и <<unlocked>> (открыт). 
    \item lights\_state --- состояние фар. Тип: строка. Может принимать одно из двух возможных значений: <<on>> (включены) и <<off>> (выключены). 
\end{enumerate}

\subsection{Код из СМС (таблица totp)}

Сущность кода из СМС содержит следующие поля.

\begin{enumerate}
    \item code --- случайный код. Тип: число. 
    \item date\_sent --- дата отправки кода. Тип: дата со временем. 
    \item date\_used --- дата использования кода. Тип: дата со временем.
    \item phone --- номер телефона, на который был отправлен код. Тип: строка. 
    \item signature --- уникальный токен. Тип: строка. 
\end{enumerate}

\section{Ограничения целостности базы данных}

Для обеспечения целостности базы данных введены следующие ограничения.

\subsection{Таблица users}

Значение столбца id является первичным ключом. По умолчанию заполняется случайным UUID.

Значение столбца status может принимать одно из трёх значений: <<pending>>, <<active>>, <<blocked>>. Не может быть пустым. По умолчанию заполняется значением <<pending>>.

Значение столбца date\_joined не может быть пустым. По умолчанию заполняется отметкой текущего времени. Дата должна быть позже, чем дата рождения (столбец birthdate).

Значение столбца phone должно быть уникальным в пределах таблицы. Не может быть пустым.

Значение столбца birthdate должно быть позже 1 января 1930 г.

\subsection{Таблица supports}

Значение столбца id является первичным ключом. По умолчанию заполняется случайным UUID.

Значение столбца middle\_name не может быть пустым.

Значение столбца last\_name не может быть пустым.

Значение столбца first\_name не может быть пустым.

Значение столбца email не может быть пустым.

Значение столбца phone должно быть уникальным в пределах таблицы. Не может быть пустым.

\subsection{Таблица technicians}

Для таблицы technicians действуют те же ограничения целостности, что и для таблицы supports.

\subsection{Таблица admins}

Для таблицы admins действуют те же ограничения целостности, что и для таблицы supports.

\subsection{Таблица scooter\_manufacturers}

Значение столбца id является первичным ключом. По умолчанию заполняется случайным UUID.

Значение столбца title не может быть пустым.

\subsection{Таблица scooter\_models}

Значение столбца id является первичным ключом. По умолчанию заполняется случайным UUID.

Значение столбца manufacturer\_id не может быть пустым. Ссылается на поле id таблицы scooter\_manufacturers.

Значение столбца title не может быть пустым.

Значение столбца single\_charge\_mileage не может быть пустым. Должно быть положительным числом.

Значение столбца weight не может быть пустым. Должно быть положительным числом.

Значение столбца max\_speed не может быть пустым. Должно быть положительным числом.

Значение столбца max\_load не может быть пустым. Должно быть положительным числом.

Значение столбца year не может быть пустым. Должно быть больше 2000.

\subsection{Таблица scooters}

Значение столбца id является первичным ключом. По умолчанию заполняется случайным UUID.

Значение столбца model\_id не может быть пустым. Ссылается на поле id таблицы scooter\_models.

Значение столбца status может принимать одно из двух значений: <<enabled>> и <<disabled>>. Не может быть пустым. По умолчанию заполняется значением <<disabled>>.

Значение столбца number не может быть пустым.

\subsection{Таблица scooters}

TODO

\subsection{Таблица pings}

TODO

\subsection{Таблица rentals}

TODO

\subsection{Таблица bookings}

TODO

\subsection{Таблица parkings}

TODO

\subsection{Таблица restricted\_zones}

TODO

\subsection{Таблица totp}

TODO

\subsection{Таблица auth\_tokens}

TODO

\subsection{Таблица settings}

TODO

\section{Описание триггера}

На рисунке~\ref{img:trigger} представлена схема алгоритма работы триггера, который не позволяет удалять пользователей, которые имеют активные (незавершенные) аренды.

\includeimage
    {trigger}
    {f}
    {H}
    {.5\textwidth}
    {Схема алгоритма работы триггера}

\section{Ролевая модель}

TODO

\section*{Вывод}

В данном разделе была спроектирована база данных для разрабатываемого приложения, а так же рассмотрены способы обеспечения ее целостности и ролевая модель.